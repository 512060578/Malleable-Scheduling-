
\documentclass{article}
\usepackage[utf8]{inputenc}
\usepackage{cite}
\usepackage{geometry}
\usepackage{listings}
\usepackage{xcolor}
\usepackage{amsmath}
\usepackage{textcomp}
\geometry{a4paper,left=3cm,right=3cm,top=2cm,bottom=2cm}
\author{}
\date{July 2020}
\title{Malleable Task Scheduling}

\begin{document}
    \maketitle
    \section{Related Work}

    The problem of malleable task scheduling has been studied for more than thirty years. Du and Leung 		 
	\cite{du1989complexity} have proved that the problem is strongly NP-hard for more than 5 processors.
	Turek \cite{turek1992approximate} has first addressed the approximation algorithms with a ratio 2,
	which was improved to $\sqrt{3}$ in \cite{mounie1999efficient} and 3/2 in \cite{mounie20023}.
	\newline
	\newline
	Several discussed the problem with precedence constraint \cite{Gunther2009Sep}\cite{Jansen2006Jul}
	\cite{Jansen2012Jan}\cite{Nishikawa2018Nov}. Du and Leung\cite{du1989complexity} proved that malleable 
	task scheduling with precedence constraints with more than 3 precessors is NP-hard. Jansen and Klaus
	\cite{Jansen2006Jul} developed an approximation algorithm with ratio 4.730598 in 2005, which deciding 
	the numbers of processors allotted to execute the tasks such that the maximum between both opposite 
	criteria of critical path length and average work in the first phase. In the second phase, it 
	generates a new allotment and to schedule all tasks according to the precedence constraints.
	The approximation ratio is improved to 3.291919 in \cite{Jansen2012Jan} which implements linear 
	program and a variant of the list scheduling algorithm and still have potentialfor further application in practice. Wu discussed the 
	problem while the task is deadline-sensitive which aims to maximize social welfare and minimize 
	machine number and the maximum weighted completion time. Dynamic programming is implemented for 
	social welfare maximization and greedy programming for social welfare maximization\cite{WuXiaohu2015AfSM}.
	\newline
	\newline
	Nishikawa and Shimada discussed Malleable Fork-Join (MFJ) tasks which
	aims to minimize the overall schedule length.\cite{Nishikawa2018Nov}\cite{Shimada2019Nov} Nishikawa apply Integer linear 
	programming (ILP) and constraint programming (CP) and the result shows that CP-based scheduling for 	
	malleable fork-join tasks can quickly obtain the better schedules than ILP-based\cite{Nishikawa2018Nov}. Shimada's algorithm 
	 improved the performance while when the system has a large number of cores
	but failed to find good results for large task graphs in a practical time\cite{Shimada2019Nov}.
	\bibliographystyle{plain}
	\bibliography{ref}
	
\end{document}


